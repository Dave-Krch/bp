% Do not forget to include Introduction
%---------------------------------------------------------------
\chapter{Úvod}
% uncomment the following line to create an unnumbered chapter
%\chapter*{Introduction}\addcontentsline{toc}{chapter}{Introduction}\markboth{Introduction}{Introduction}
%---------------------------------------------------------------
\setcounter{page}{1}

% The following environment can be used as a mini-introduction for a chapter. Use that any way it pleases you (or comment it out). It can contain, for instance, a summary of the chapter. Or, there can be a quotation.
 %\begin{chapterabstract}
%	\lipsum[1]
%\end{chapterabstract}

%\lipsum[2][1-4]{} [1]

%\lipsum[4]

V2X (Vehicle-to-everything) komunikace je důležitá technologie pro budoucnost
dopravního průmyslu. Hlavními cíli V2X je zvýšit bezpočnost, efektivitu
a umožnit tak udržitelný růst silničního provozu. Vedlejším, ale neméně důležitým
cílem, je také snížení dopadu dopravy na životní prostředí, právě zvýšením
efektivity.
Součástí je množství standardů popisujících komunikaci mezi různými zařízenímí
(vozidlo – vozidlo, vozdilo – zařízení infrastruktury) a různé V2X zprávy.
Moje práce se zaměřuje hlavně na zprávy CAM [1] (Coop awarenes message)
a DENM [2] (Decentralized enviromental message).
Komunikace V2X i standardy používané pro její kódování (ASN.1) existují
již řadu let, proto jsou již dostupné různé komerční i opensource překladače
a dekodéry. Hlavním cílem mé práce je porovnat dostupná řešení, vybrat z nich
nejlépe udržované a dobře použitelné projekty, hlavně ASN.1 kompilátor, a s jejich
pomocí vytvořit aplikaci pro dekódování a vizualizaci CAM a DENM zpráv.
Moje motivace pro dané téma vychází z několika důvodů. Prvním jsou použité
technologie, s prací v prostředí .NET mám již kladné zkušenosti a chci
se jí tak věnovat v budoucnu. Dalším důvodem je, že vyvíjená aplikace může
mít hned po dokončení práce reálné využití. Posledním důvodem je budoucnost
řešené problematiky, podle mého názoru bude V2X komunikace v blízké
budoucnosti jedině růst a práce tak zůstane aktuální.

%===============================================================================
\chapter{Cíl práce}

Cílem práce je cíl práce


%===============================================================================
\chapter{V2X}
Krátký popis a smysl v2x

\section{Historie}
Historie v2x

\section{Současné využití V2X}


\section{Druhy V2X zpráv}

\subsection{CAM}

\subsection{DENM}

%===============================================================================
\chapter{ASN.1}
Krátký popis

\section{Využití}

\section{Základní struktura}

\section{Parametrizace}
Krátce, v práci se nepoužívá

\section{Informační objekty}
Krátce, v práci se nepoužívá

\section{Pravidla kódování}
Popsat hlavně per, možná ber a další jen zmínit?

\subsection{BER}

\subsection{PER}

\subsection{Ostatní pravidla}

%===============================================================================
\chapter{Kompilátor}

%===============================================================================
\chapter{Dekodér}

%===============================================================================
\chapter{UI aplikace}