% Do not forget to include Introduction
%---------------------------------------------------------------
\chapter{Úvod}
% uncomment the following line to create an unnumbered chapter
%\chapter*{Introduction}\addcontentsline{toc}{chapter}{Introduction}\markboth{Introduction}{Introduction}
%---------------------------------------------------------------
\setcounter{page}{1}

% The following environment can be used as a mini-introduction for a chapter. Use that any way it pleases you (or comment it out). It can contain, for instance, a summary of the chapter. Or, there can be a quotation.
 %\begin{chapterabstract}
%	\lipsum[1]
%\end{chapterabstract}

%\lipsum[2][1-4]{} [1]

%\lipsum[4]

Se zvyšujícím se počtem obyvatel Země se zvyšuje i poptávka po mobilitě, což vede k nárůstu množství dopravních prostředků.
Větší hustota dopravy sebou ale přináší nevýhody. S rostoucím počtem dopravních prostředků stoupá i množství nehod, klesá rychlost a efektivita dopravy a s tím je spojené i větší znečištění životního prostředí.

Proto je důležité vyvíjet a využívat technologie, které se těmito problémy zabývají. Jednou z těchto technologií je právě V2X (Vehicle-to-everything) komunikace.
Hlavními cíli V2X je zvýšit bezpečnost, efektivitu
a umožnit tak udržitelný růst silničního provozu. Vedlejším, ale neméně důležitým
cílem, je také snížení dopadu dopravy na životní prostředí, právě zvýšením
efektivity.

Součástí V2X jsou různé standardy popisující komunikaci mezi několika druhy zařízení
(vozidlo – vozidlo, vozidlo – zařízení infrastruktury) a různé V2X zprávy.
Moje práce se zaměřuje hlavně na zprávy CAM (Coop awarenes message)
a DENM (Decentralized enviromental message).

V2X zprávy jsou kódované pomocí Abstract Syntax Notation One (ASN.1), nástroje pro kódování abstraktních datových struktur. Na překlad ASN.1 specifikací a dekódování V2X zpráv se soustředí první část mojí práce.

Komunikace V2X i standardy používané pro její kódování (ASN.1) existují
již řadu let, proto jsou již dostupné různé komerční i opensource překladače
a dekodéry. Hlavním cílem mé práce je porovnat dostupná řešení, vybrat z nich
nejlépe udržované a dobře použitelné projekty, hlavně ASN.1 kompilátor, a s jejich
pomocí vytvořit aplikaci pro dekódování a vizualizaci CAM a DENM zpráv.

Moje motivace pro dané téma vychází z několika důvodů. Prvním jsou použité
technologie, s prací v prostředí .NET mám již kladné zkušenosti a chci
se jí tak věnovat v budoucnu. Dalším důvodem je, že vyvíjená aplikace může
mít hned po dokončení práce reálné využití. Posledním důvodem je budoucnost
řešené problematiky, podle mého názoru bude V2X komunikace v blízké
budoucnosti jedině růst a práce tak zůstane aktuální.

%===============================================================================
\chapter{Cíl práce}

Tato bakalářská práce se dá rozdělit na několik navazujících částí.

Prvním cílem mojí práce, je navrhnout architekturu v prostředí .NET, která bude vhodná pro překlad a dekódování. Architekturou je myšlena struktura a rozvržení C# tříd, které budou odpovídat zadaným ASN.1 specifikacím a budou generované kompilátorem.

Dalším cílem je nalezení nebo vytvoření programu, který bude překládat ASN.1 specifikace do zdrojového kódu v jazyce C#. Psaní celého kompilátoru není záměrem práce, proto se v této části zabývám nejprve výběrem vhodného kompilátoru, schopného zpracovávat potřebné ASN.1 specifikace pro zadané V2X zprávy a poté případným dopsáním funkcí nebo jeho úpravou, tak, aby výsledný program generoval zdrojový kód v jazyce C#.

Třetí část práce se zaměřuje na dekódování ASN.1 zpráv. Tato část přímo navazuje na část první, dekodér pro práci využívá zdrojový kód vygenerovaný kompilátorem.

Práce se zaměřuje hlavně na zprávy ze standardů V2X, ale kompilátor i dekodér lze použít na jakékoliv specifikace, které používají stejné principy z ASN.1.

Poslední cíl práce je návrh a implementace UI aplikace.




%===============================================================================
\chapter{V2X}
Krátký popis a smysl v2x

\section{Historie}
Historie v2x

\section{Současné využití V2X}


\section{Druhy V2X zpráv}

\subsection{CAM}

\subsection{DENM}

%===============================================================================
\chapter{ASN.1}
Krátký popis

\section{Využití}

\section{Základní struktura}

\section{Parametrizace}
Krátce, v práci se nepoužívá

\section{Informační objekty}
Krátce, v práci se nepoužívá

\section{Pravidla kódování}
Popsat hlavně per, možná ber a další jen zmínit?

\subsection{BER}

\subsection{PER}

%===============================================================================
\chapter{Kompilátor}

%===============================================================================
\chapter{Dekodér}

%===============================================================================
\chapter{UI aplikace}

%===============================================================================
\chapter{Ekonomicko-manažerské zhodnocení}

%===============================================================================
\chapter{Závěr}